\documentclass[10pt]{scrartcl}
\usepackage[T1]{fontenc}
\usepackage[utf8]{inputenc}
\usepackage{psfrag,amsmath,amsfonts,verbatim,mathtools}
%\usepackage[margin=.85in]{geometry}
%\usepackage{fullpage}
\usepackage[parfill]{parskip}
\usepackage{bookman}
\usepackage[small,bf]{caption}
\usepackage[shortlabels]{enumitem}
\usepackage{xfrac}
\usepackage{commath}
\usepackage{tensor}
\usepackage{titlesec}
\usepackage{physics}
\usepackage{courier}
\usepackage[hang,ragged]{footmisc} 
\usepackage{caption}
\newenvironment{sysmatrix}[1]
 {\left[\begin{array}{@{}#1@{}}}
 {\end{array}\right]}
\newcommand{\ro}[1]{%
  \xrightarrow{\mathmakebox[\rowidth]{#1}}%
}
\newlength{\rowidth}
\AtBeginDocument{\setlength{\rowidth}{3em}}

\newcommand{\ones}{\mathbf 1}
\newcommand{\reals}{{\mbox{\bfseries R}}}
\newcommand{\integers}{{\mbox{\bf Z}}}
\newcommand{\symm}{{\mbox{\bf S}}} 
\newcommand{\Mod}[1]{\ (\mathrm{mod}\ #1)}

\newcommand{\nullspace}{{\mathcal N}}
\newcommand{\range}{{\mathcal R}}
\newcommand{\Rank}{\mathop{\bf Rank}}
\newcommand{\diag}{\mathop{\bf diag}}
\newcommand{\card}{\mathop{\bf card}}
\newcommand{\proj}{\mathop{\bf Proj}}
\newcommand{\conv}{\mathop{\bf conv}}
\newcommand{\zero}{\mathop{\bf 0}}
\newcommand{\prox}{\mathbf{prox}}

\newcommand{\Expect}{\mathop{\bf E{}}}
\newcommand{\Prob}{\mathop{\bf Prob}}
\newcommand{\Co}{{\mathop {\bf Co}}} 
\newcommand{\dist}{\mathop{\bf dist{}}}
\newcommand{\argmin}{\mathop{\rm argmin}}
\newcommand{\argmax}{\mathop{\rm argmax}}
\newcommand{\epi}{\mathop{\bf epi}} 
\newcommand{\Vol}{\mathop{\bf vol}}
\newcommand{\dom}{\mathop{\bf dom}} 
\newcommand{\intr}{\mathop{\bf int}}
\newcommand{\sign}{\mathop{\bf sign}}

\newcommand{\cf}{{\it cf.}}
\newcommand{\eg}{{\textit e.g.}}
\newcommand{\ie}{{\textit i.e.}}
\newcommand{\etc}{{\it etc.}}
\bibliographystyle{alpha}





\usepackage{titling}
%\setlength{\droptitle}{-2cm}
\nonfrenchspacing


\title{Homework 4 for Physics  503}
\author{Ashok M. Rao}

\begin{document}
\maketitle




\paragraph{Problem 3}

A rudimentary action for a particle moving in a Lorentz field $A_{\mu}(x)$ is given by
\begin{align}
S &= \int \left\{-m\sqrt{-\eta_{\mu\nu}\dd{x}^\mu\dd{x}^\nu}+A_{\mu}(x)\dd{x}^\mu \right\}\\ &= \int \dd{\tau} A_{\mu}(x(\tau))\dv{x^\mu}{\tau} - m\int\dd{\tau}\sqrt{-\eta_{\mu\nu}\dv{x^\mu}{\tau}\dv{x^{\nu}}{\tau}} 
\end{align} 
The variational solution to this leads to the field strength tensor 
\begin{align*}
	F_{\mu\nu}(x)= \partial_\mu A_\nu(x) - \partial_\nu A_\mu (x)
\end{align*}
giving the particle's equation of motion $m\dv[2]{x^{\mu}}{\tau}-\tensor{F}{^\mu_\nu}\dv{x^{\nu}}{\tau} = 0	$ wherein $\tensor{F}{^\mu_\nu} =\eta^{\mu\rho}F_{\rho\nu}$. \footnote{ 
To see this, calculate the variation for either term on the last equation in (1). For the former term, it is straightforward to see that $\var(-m\int \dd{\tau}\sqrt{-\eta_{\mu}{\nu}\dv*{x^\mu}{\tau}\dv*{x^{\nu}}{\tau}}) = \int\dd{\tau}\left[-m \eta_{\mu\nu}\dv*[2]{x^{\mu}}{\tau}\var{x^\nu}\right]$. Pull the variation through on the latter integrating by parts concluding $\var{\dv{x^{\nu}}{\tau} + \int \dd{\tau}A_{\mu}(x(\tau))\dv{x^\mu}{\tau}}=\int \dd{\tau}(\partial_\mu A_\nu - \partial_\nu A_\mu)\dv{x^{\nu}}{\tau}\var{x^{\mu}}$. The field tensor as stated above is clearly evident in this latter term. This gives the solution to (1) as $\var{s} = \int \dd{\tau}\left\{-m\eta_{\mu\sigma}\dv[2]{x^{\mu}}{\tau}\var{x^{\sigma}} + F_{\mu\nu}\dv{x^{\nu}}{\tau}\var{x^{\mu}}\right\} = \int \dd{\tau}\left\{-m\dv[2]{x^{\mu}}{\tau} + \left(\eta^{\mu\rho}F_{\rho\nu}\right)\dv{x^{\nu}}{\tau}\right\}\eta_{\mu\sigma}\var{x^{\sigma}}$, and from this the equation of motion follows directly.}  The field strength tensor is due to the $\int A_{\mu}(x)\dd{x}^\mu$ in (1) and is responsible for describing the field generated by the particles. The solution is overdetermined in that a gauge transformation, $A_\mu \rightarrow \tilde{A}_{\mu}= A_\mu + \partial_\mu \chi$ leaves the action alone.\footnote{In particular, $\int \partial_\mu \chi(x)\dd{x}^\mu = \int_{\tau_0}^{\tau} \dd{\tau}\dv*{\chi(x)}{\tau} = \chi(x(\tau))-\chi(x(\tau_0))$. This is evident in that $\partial_\mu A_\nu(x) \rightarrow\partial_\mu\tilde{A}_{\mu} = F_{\mu\nu} + \partial_\mu\partial_\nu \chi - \partial\nu\partial\mu\chi = F_{\mu\nu}$ as expected.}  This symmetry due to the field strength suggests that (1) should be appended with a gauge invariant object, the simplest of which would be $F^{\mu\nu}F_{\mu\nu}$. Hence we have the additional variation,
\begin{align}
\var{\int \dd[4]{x}\left(F^{\mu\nu} F_{\mu\nu}\right)}	 = -2\int\dd[4]{x} \var(F^{\mu\nu}{F_{\mu\nu}}) &= -4\int\dd[4]{x}\partial_{\mu}F^{\mu\nu}\var{A_\nu(x)}
\end{align}
Extremizing, obtain that $\partial_\mu F^{\mu\nu} = 0$. Note also that going forward the action will be modified by a factor so that the functional gives unit integrand in (3). To complete the action stimulated by (1), the former terms should sum over many charged particles. That is we have,
\begin{align}
S = \int\sum_{(x, e)\in\mathcal{A}}\left\{-m_a\sqrt{-\eta_{\mu\nu}\dd{x}^\mu \dd{x}^{\nu}}+e A_{\mu}(x, t)\dd{x}^{\mu}\right\}-\int\frac{1}{4} \dd[4]{x} F^{\mu\nu}F_{\mu\nu}
\end{align}
Here $\mathcal{A}$ is just any countable bag of particles $x$ with their associated charge $q$.  The second term in the sum is the only non-trivial change and so to solve the entire thing in the form of (2), we can solve an equivalent action where $A_{\mu}(x)$ may be varied by pulling out a $\delta$ function:
\begin{align}
	\var(S_{current})\rightarrow \int\dd[4]{x} \left[\sum_{i\in\mathcal{A}}e_i \int\dd{\tau_i} \delta^{i}_{j}(x-q_i(\tau_i))
 \dv{{q}_{i}^{\mu}}{\tau_i}\right] \var{A_{\mu}(x, t)}
\end{align}
The bracketed term is just the current, $\eg$ $J^{\mu}(x)\equiv \sum_{i\in\mathcal{A}}e_i \int\dd{\tau_i} \delta^{i}_{j}(x-q_i(\tau_i)) \dv{{q}_{i}^{\mu}}{\tau_i}$. Putting together (4) and (5),
 \begin{align}
 \partial_{\mu}F^{\mu\nu}(x) = -J^{\nu}(x)\rightarrow \partial_\nu\partial_\mu F^{\mu\nu}(x) = - \partial_\nu J^{\nu}(x)\rightarrow \partial_\nu J^{\nu}=0
 \end{align}
The final result in the chain above is due to anticommutativity of the field tensor, and implies conservation of current. Given standard identification of the field tensor elements\footnote{Like this, though noting that matrix elements are counted from 0 given the time interaction  \[F_{\mu\nu} \equiv \partial_\mu A_\nu - \partial_\nu A_\mu = \smqty(0& - E_x& -E_y& -E_z\\E_x&0B_z&-B_y\\E_y&-B_z&0&B_z\\E_z&B_y&-B_z&0)\]
}, obtain the vector Maxwell's equations:
\begin{align}
 \div{\va{E}} = \rho \qc \curl{\va{B}} = \pdv{\va{E}}{t} + \va{J} 
\end{align}
which follow directly from (7) for $\nu = 1, 3$ respectively. $F_{\mu\nu}$ is totally antisymmetric and so let $F_{\gamma\rho} = \partial_{\gamma}A_{\rho} - \partial_{\rho}A_{\gamma}$ write out that $\epsilon^{\mu\nu\gamma\rho}\partial_{\nu}F_{\gamma\rho} = 	\epsilon^{\mu\nu\gamma\rho}(\partial_\nu \partial_\gamma A_\rho - \partial_\nu\partial_\rho A_\gamma) = \epsilon^{\mu\nu\gamma\rho}\partial_\nu (\partial_\gamma A_\rho - \partial_\rho A_\gamma)$. Given anticommutativity, this reduces to $2\epsilon^{\mu\nu\gamma\rho}\partial_\nu\partial_\gamma A_\rho = 0$. Like (7), this implies for $\mu=3, 0$ respectively that,
\begin{align}
\curl{\va{E}} = -\pdv{\va{B}}{t} \qc \div{\va{B}} = 0	
\end{align}
Given that the electromagnetic potential $A=(\phi, A)$ is a one-form the exterior derivative $F=\dd{A}$ must be the field 2-form. To see that explicitly,
\begin{align}
\dd{A} &= (1/1!)\partial_\nu A_\mu \dd{x}^{\nu}\wedge\dd{x}^{\mu}	= (1/2)(\partial_\nu A_\mu - \partial_\mu A_\nu)\dd{x}^\nu \wedge \dd{x}^\mu 
\end{align}
Thus we clearly have that $F = (1/2!)F_{\mu\nu}\dd{x}^{\mu}\dd{x}^{\nu}$, directly obtaining $F_{\mu\nu}$ as an exterior derivative on the assigned potential from the definition. To discover that $\dd{\dd{A}}=0$ directly from the definition, calculate that
\begin{align}
\dd{\dd{A}} &=(1/1!)\partial_\gamma(\partial_\nu A_\mu)\dd{x}^{\gamma}\wedge\dd{x}^{\nu}\wedge\dd{x}^\mu \\ &=(1/2)\partial_\gamma(\partial_\nu A_\mu - \partial_\mu A_\nu)\dd{x}^{\gamma}\wedge\dd{x}^{\nu}\wedge\dd{x}^\mu 
\end{align}
Partial derivatives commute and thus (11) is simultaneously forced to satisfy both $\partial_{\gamma}\partial_{\nu}-\partial_{\nu}\partial_{\gamma} = 0$ and $\dd{x}^{\gamma}\wedge\dd{x}^{\nu} + \dd{x}^{\nu}\wedge\dd{x}^{\gamma} = 0$. This leaves no choice but to conclude that the RHS of (11) vanishes such that $\dd{F} = \dd{\dd{A}} = 0$. Note that,
\begin{align}
\va{E} &= -\left(\grad{\phi} + \pdv{t}A(x, t)\right)\qc \va{B} = \curl{\va{A}}	
\end{align}
Applying the identity $\dd{\dd{A}}=0$ would re-derive the two Maxwell equations in (8). The conclusions are analogous, with each hinging on both the very trivial commutativity of partial derivatives as well as the anticommutativity of ``curl"-like derivations.  The Hodge dual of the field strength tensor is $\ast F_{\mu\nu} = \frac{1}{2} F^{\kappa\lambda}\epsilon_{\kappa\lambda\mu\nu}$. Calculating the proposed action, 
\begin{align}
S &= \frac{1}{2}\int \left(\frac{1}{2!}F_{\mu\nu}\dd{x^{\mu}}\dd{x^{\nu}}\right) \wedge \left(F^{\kappa\lambda} \epsilon_{\kappa\lambda\mu\nu}\dd{x^{\kappa}}\dd{x^{\lambda}}\right)\\
&= -\frac{1}{4}\int \left(F_{\mu\nu}\dd{x^{\mu}}\dd{x^{\nu}}\right) \wedge \left(F^{\mu\nu}\delta_{\mu\nu}\dd[2]{x}\right) \\
&= -\frac{1}{4}\int F_{\mu\nu}F^{\mu\nu} (\dd{x^{\mu}}\dd{x^{\nu}}\wedge \delta_{\mu\nu}\dd[2]{x})\\
&= -\frac{1}{4}\int F_{\mu\nu}F^{\mu\nu}\dd[4]{x}
\end{align}
This is matches the previously derived action corresponding to the Lagrangian density with $\mathcal{L} = - \frac{1}{4}F^{\mu\nu}F_{\mu\nu}$. Now expanding, $F\wedge F = \frac{1}{4}(F_{\mu\nu}\dd{x^\mu}\wedge\dd{x^{\nu}})	(F_{\kappa\lambda}\dd{x^\kappa}\wedge\dd{x^{\lambda}})= (-1)^{p^2}(F\wedge F)$
Thus for odd $p$, like $p=3$, it must be that $(F\wedge F) = 0$ and $\omega$ is a closed form. If we are in flat space, all closed forms are exact and so there exists a vector potential $A$ such that $\omega=\dd{A}$. Calculate that in spacetime, with $d=4$, 
\begin{align}
\omega &= \int(\frac{1}{2!}F_{\mu\nu}\dd{x^\mu}\wedge\dd{x^{\nu}})	(\frac{1}{2!}F_{\kappa\lambda}\dd{x^\kappa}\wedge\dd{x^{\lambda}})\\
&= \int \dd[4]{x}\epsilon^{\mu\nu\kappa\lambda}\frac{1}{2!}F_{\mu\nu}\frac{1}{2!}F_{\kappa\lambda}\\
&= \int \dd[4]{x}\epsilon^{\mu\nu\kappa\lambda}(\partial_\mu A_\nu)(\partial_\kappa A_\lambda)\\
&= \int \dd[4]{x}\partial_\mu(\epsilon^{\mu\nu\kappa\lambda}A_\nu\partial_\kappa A_\lambda)
\end{align}
Up to a constant (and possibly not the one I have), the integrand has the value of a total divergence, $\partial_\mu(\epsilon^{\mu\nu\kappa\lambda} A_\nu \partial_\kappa A_\lambda)$. Given that the integral is over Minkowski space, I would imagine the result is a 4-form (unless I am missing something). However, over $\mathbb{R}^3\subset \mathcal{M}$, the analogous integral would take the form of Carroll (2.90): $\int_{\mathbb{R}^3}:\omega\mapsto \mathbb{R}$ which is a 3-form. Solving this,
\begin{align}
S = C\int \dd[3]{x}\epsilon^{\mu\nu\kappa} A_{\mu} F_{\nu\kappa}	 = C\int \dd[3]{x}\epsilon^{\mu\nu\kappa}A_\mu \partial_{\nu} A_\kappa \rightarrow C\int A\dd{A}
\end{align}
The $C$ is just to indicate a constant, I don't know what it is.

\paragraph{Problem 5} The general expression for the Christoffel symbol is,
\begin{align}
	\Gamma_{\mu\nu}^{\lambda} &= \frac{1}{2}g^{\lambda\sigma}(\partial_{\mu}g_{\nu\sigma}+\partial_\nu g_{\sigma\mu} - \partial_{\sigma}g_{\mu\nu})\\
	&= 0
\end{align}
given that no diagonal component could be summed over. Now, going through each component at a time,
\begin{align}
	\Gamma_{\mu\mu}^{\lambda} &= \frac{1}{2}g^{\lambda\sigma}(\partial_{\mu}g_{\mu\sigma}+\partial_\mu g_{\sigma\mu} - \partial_{\sigma}g_{\mu\mu})\\
	 &= \frac{1}{2}g^{\lambda\lambda}(\partial_{\mu}g_{\mu\lambda}+\partial_\mu g_{\lambda\mu} - \partial_{\lambda}g_{\mu\mu})\\
	 &= -\frac{1}{2}(g_{\lambda\lambda})^{-1}\partial_{\lambda}g_{\mu\mu}\\
	\Gamma_{\mu\lambda}^{\lambda} &= \frac{1}{2}g^{\lambda\lambda}(\partial_{\mu}g_{\lambda\lambda}+\partial_\lambda g_{\lambda\mu} - \partial_{\lambda}g_{\mu\lambda})\\
	&= \frac{1}{2}(g_{\lambda\lambda})^{-1}\partial_{\mu}g_{\lambda\lambda}\\
	&= \partial_{\mu}(\log\sqrt{\abs{g_{\lambda\lambda}}})\\
\Gamma_{\lambda\lambda}^{\lambda} &= \frac{1}{2}g^{\lambda\lambda}(\partial_{\lambda}g_{\lambda\lambda}+\partial_\lambda g_{\lambda\lambda} - \partial_{\lambda}g_{\lambda\lambda})\\
&= \partial_{\lambda}(\log{\sqrt{g_{\lambda\lambda}}})
\end{align}

Next, to solve the geodesic equation in the equatorial plane, letting $\xi$ be our affine parameter into $\dv[2]{x^{\lambda}}{\tau} + \Gamma^{\lambda}{\mu\nu}\dv{x^{\mu}}{\tau}\dv{x^{\nu}}{\tau}$. Starting with $\Gamma^{\theta}_{\phi\phi}$ let's us being making the most of symmetry,
\begin{align}
\dv[2]{\theta}{\tau} - \sin\theta\cos\theta\left(\dv{\phi}{\tau}\right)^2 = 0	
\end{align}
This is solved by letting $\theta(\tau)=\theta=\frac{\pi}{2}$. Continuing,
\begin{align}
&\left(1-\frac{2M}{r}\right)\left[\dv[2]{t}{\tau} + \dv{t}{\tau}\dv{\tau}\right] = 0 \qcomma r^2\dv[2]{\phi}{\tau} + \dv{\tau}\dv{\phi}{\tau}r^2 = 0,\qcomma \dv{\tau}\left(r^2\dv{\phi}{\tau}\right)=0
\end{align}
Given the extent of symmetry, the others are either unnecessary or may be trivially solved. For example, for the first equation in (33) letting $\dv{t}{\tau} \propto \left(1-\frac{2M}{r}\right)^{-1}$ is a solution. With $\sin{\theta}=\pi/2$ the same can be said for the rightmost equation. Thus,
\begin{align}
\dv{t}{\tau} &= \frac{k_1}{1-\frac{2M}{r}}\\
\dv{\phi}{\tau} &= \frac{k_2}{r^2}	
\end{align}
Instead of further solving second-order equations, we can without loss of generality substitute what we have into
\begin{align}
		-g_{tt}\left(\dv{t}{\tau}\right)^2 +g_{rr}\left(\dv{r}{\tau}\right)^2+r^2\left(\dv{\phi}{\tau}\right)^2 = 1
\end{align}
which holds by definition. Now substituting (34) and (35), 
\begin{align}
	\frac{1}{1-\frac{2M}{r}}\left[k_1^2 - \left(\dv{r}{\tau}\right)^2\right]  = 1+\frac{k_2^2}{r^2}
\end{align}
Rearranging,
\begin{align}
\frac{1}{2}\left(\dv{r}{\tau}\right)^2 + \left[\frac{1}{2}\left(1-\frac{2M}{r}\right)\left(1+\frac{k_2^2}{r^2}\right)- \frac{k_1^2}{2}\right] = 0
\end{align}
This has the shape of a classical potential.  Expanding the potential (the term inside the bracket) and giving more convenient names to the integration constants obtain,
\begin{align}
V(r) &= k\left(\frac{1}{2} - \frac{M}{r}\right) + \frac{\omega^2}{r^2}\left(\frac{1}{2}-\frac{M}{r}\right)	
\end{align}
Giving the general solution (under the symmetries pursued)
\begin{align}
\frac{1}{2}\left(\dv{r}{\tau}\right)^2 + \left[k\left(\frac{1}{2} - \frac{M}{r}\right) + \frac{\omega^2}{r^2}\left(\frac{1}{2}-\frac{M}{r}\right)\right] = 0	
\end{align}


\paragraph{Problem 2} The given metric appears to correspond to a flat metric that ``entertains" two timeline coordinates, that is with the signature $(-1, -1, 1, 1, 1)$:
\begin{align}
ds^2 &= -\left(X_0^2 + X_4^2\right) + X_1^2 + X_2^2 + X_3^2	
\end{align}
This might be realized with a block matrix (where perhaps the top-left block has unit negative determinant). Suppose the following embedding,
\begin{align}
	X_0 &= \sqrt{1+r^2}\cos{t} \\
	X_1 &= r\sin{\theta}\cos{\phi}\\
	X_3 &= r\sin{\theta}\sin{\phi}\\
	X_2 &= r\cos{\phi}\\
	X_4	&= \sqrt{1+r^2}\sin{t}
\end{align}
The motivation for this embedding is unsophisticated. Given the diagonality of the metric, the spacelike coordinates should look spherical. Initially, we may suppose the timelike coordinates are independently polar, say with radius T.  However, the embedded hypersurface results in a constraint and the ``time radius" cannot be independent of the spatial radius. Particularly, we have that $T^2 - r^2 = - l^2$ and differentiating results in the coordinates given above.  

Placing parenthesis around the hypersurface equation given leads to another change of coordinates. First express the surface as
\begin{align}
	(X_0^2 - X_1^2) + (X_4^2 - X_3^2) - X_2^2 = \ell^2
\end{align}
(Notice the sign flip). Define,
\begin{align}
	(X_0^2 - X_1^2) &= \frac{t^2 - x^2}{\omega^2}\\
	(X_4^2 - X_3^2) &= \ell^2 - \frac{t^2-x^2}{\omega^2}
\end{align}
And now apply the following change of coordinates,
\begin{align}
&X_0\mapsto t/\omega\qcomma X_1\mapsto x/\omega\qcomma X_4\mapsto y/\omega	\\
&L_+ \mapsto (1/\omega)(x^2 + y^2 - t^2)\qcomma L_- \mapsto 1/\omega 	
\end{align}
Now give back the metric as 
\begin{align}
 \bar{ds}^2 &= (\ell/\omega)^2(-dt^2 + dx^2 + dy^2 + dz^2 + d\omega^2) 
\end{align}
The point of this was to express this as a conformal mapping of Minkowski, so that $ds^2 = 0 \iff \bar{ds}^2 = 0$. Thus a lightlike trajectory will follow
\begin{align}
d\omega^2 = dt^2 - (dx^2 + dy^2 + dz^2)	
\end{align}
As such, if a light beam is directed towards the $\omega=0$ boundary at some observer's coordinate $\omega_O$, it will always return in a fixed time interval (independent of ``distance" traveled) so long as there is a reflecting surface. Correspondingly, if we modify (52) so that $\omega\mapsto (\ell^2/r)$, the metric changes to
\begin{align}
ds^2 &= (r/\ell)^2(-dt^2 + dx^2 + dy^2 + dz^2) + (\ell/r)^2dr^2	
\end{align}

Naturally, $r\to\infty$ corresponds to $\omega\to 0$.  Up to constant factors due to $\ell$, in this limit the metric approaches,
\begin{align}
	ds^2\sim r^2(dx^2 + dy^2 + dz^2 - dt^2)
\end{align}
Hence, the ``boundary" of the space must be reached in finite-only coordinate time. 

\paragraph{Problem 1}
This is a diagonal metric and so has the coefficients as in Problem 5. Three don't vanish,
\begin{align}
\Gamma_{zt}^{t} &= \partial_z(\log\sqrt{\abs{1 + a z}})=\frac{a}{2(1+a z)}\\
\Gamma_{tt}^{t} &= \partial_t(\log\sqrt{\abs{1 + a z}})= \frac{1}{2}\frac{a\pdv*{z}{t}}{1 + a z}\\
\Gamma_{tt}^{z} &= -(1/2)\partial_z [-(1+a z)] = a/2
\end{align}
Solve that,
\begin{align}
\dv[2]{z}{\tau}	 &= -\frac{a}{2}\left(\dv{t}{\tau}\right)^2= -\frac{a}{2(1+v^2)}\simeq -\frac{a}{2}\left(1 + v^2 + \mathcal{O}(v^4)\right)
\end{align}
where we have used that $\dv{t}{\tau}=\gamma$. To obtain a locally inertial coordinate system, 
\begin{align}
x^{\mu} &= \dv{x^{\mu}}{x^{\lambda}}\{x^{\lambda}	+ \frac{1}{2}\Gamma_{\mu\nu}^{\lambda}x'^{\lambda} x'^{\nu}\}\\
&=  \tensor{\eta}{^\mu_\lambda}\{x'^{\lambda}	+ \frac{1}{2}\Gamma_{\mu\nu}^{\lambda}x'^{\lambda} x'^{\nu}\}\\
&= \tensor{\eta}{^\mu_\lambda}x'^{\lambda} -\frac{1}{2}\Gamma_{\lambda\nu}^{\mu} x'^{\lambda} x'^{\nu}
\end{align}
Simplifying even further, we would have $x^{\mu} = x'^{\mu} - \frac{1}{2}\Gamma_{\lambda\nu}^{\mu}x'^{\nu}x'^{\lambda}$. This is a coordinate map onto a locally flat space obtained by reverse engineering the Christoffel coefficients. Within the transformed frame, all non-flatness in the metric is of second or higher order. The geodesic equation in (59) appears analogous to the Poisson equation for gravity, and \eg  $\frac{2V}{m}\sim a z$.  The coordinate system disagrees with the general metric (at second or higher order) only on measurements of $(t, z)$, suggesting that there are certain flat directions but not others (due to the field, maybe). 
\end{document}